\documentclass{article}

\usepackage[utf8]{inputenc}
\usepackage[finnish]{babel}
\usepackage{hyperref}

\setlength{\parindent}{0.0in}
\setlength{\parskip}{0.1in}

\begin{document}
\title{Joululabra 2012 käyttöohje - Sokoban}
\author{Mika Viinamäki}
\maketitle

\section*{Käyttöohje}
Sokobanin saa käynnistettyä purkamalla GitHub-repositoryn juuressa olevan zip-tiedoston jonnekin ja ajamalla zip-tiedostossa olevan \verb=run.sh= (Linux, OSX) tai \verb=run.bat= (Windows) -tiedoston. Toimivuus on testattu Linuxilla ja Windowsilla - Windowsilla bat-skriptin toimiminen vaatii, että \verb=java=-ohjelma löytyy PATH-ympäristömuuttujasta.

Valikoissa liikkuminen toimii nuolinäppäimillä ja valinta tehdään enter-painikkeella. Escape-painikkeella pääsee liikkumaan valikoissa taaksepäin. Lisäksi itse peliruudussa kentän voi aloittaa alusta R-painikkeella.

\end{document}
