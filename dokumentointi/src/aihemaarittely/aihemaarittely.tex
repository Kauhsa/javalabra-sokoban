\documentclass{article}

\usepackage[utf8]{inputenc}
\usepackage[finnish]{babel}
\usepackage{hyperref}

\setlength{\parindent}{0.0in}
\setlength{\parskip}{0.1in}

\begin{document}
\title{Joululabra 2012 aihemäärittely - Sokoban}
\author{Mika Viinamäki}
\maketitle

\section*{Aiheen kuvaus}
Sokoban on melko yksinkertainen, ruudukkopohjainen puzzle-peli, jossa ruudukossa sijaitsevaa pelaajaa liikuttaen on tarkoitus työntää kaikki laatikot tiettyyn kohtaan ruudukkoa. Ruudukossa sijaitsee seiniä, joiden läpi ei voi liikkua tai työntää laatikoita. Pelaaja ei myöskään voi työntää kahta laatikkoa yhtäaikaa.
    
Pelistä löytyy kunnollinen kuvaus wikipediasta: \url{http://en.wikipedia.org/wiki/Sokoban}
    
\section*{Ominaisuudet}
Projekti koostuu muutamasta osiosta, joita voisi ajatella omiksi kokonaisuuksikseen:

\subsection*{Sokoban-engine}
Projektissa on toteutettu melko joustava Sokoban-engine, joka mahdollistaa monenlaisten erityyppisten objektien toteuttamisen helposti - esimerkkinä mainittakoon esimerkiksi kivet, joita voi työntää useamman kerralla tai vaikkapa lattia, johon ei voi työntää kiviä.

Toteutettuna ei kuitenkaan ole kuin perus-Sokobanin eri objektit - olemassa olevan koodin pyörittelyyn ja viilaamiseen tuntui kuluvan pelottavan paljon aikaa ja päätin olla hamstraamatta ylimääräisiä ominaisuuksia.

Engine tukee myös useampaa pelaajaobjektia, joista pelaaja voi valita liikuttamansa pelaajan, mutta tätä ei ole toteutettu käyttöliittymässä.

\subsection*{YAML-lataaja}
Sokoban-engineen voi ladata kenttiä YAML-muotoisista tiedostoista. Kenttäformaatissa on pieniä rajoituksia - esimerkiksi laatikoiden maaleja ei voi laittaa alkutilanteessa laatikoiden alle.

\subsection*{Graafinen käyttöliittymä}
Pelin graafinen käyttöliittymä on toteutettu Slickillä. Suurin osa käyttöliittymäkoodista koostuu erilaisista käyttöliittymäelementeistä - esimerkiksi valikko ja "label", joita voi hyvin käyttää tämän projektin ulkopuolellakin. Käyttöliittymäelementit ovat melko monipuolisia, ja tukevat esimerkiksi keskittämistä mihin tahansa reunaan pysty- ja vaakatasossa.

\end{document}
