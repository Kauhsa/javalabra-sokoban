\documentclass{article}

\usepackage[utf8]{inputenc}
\usepackage[finnish]{babel}
\usepackage{hyperref}

\setlength{\parindent}{0.0in}
\setlength{\parskip}{0.1in}

\begin{document}
\title{Joululabra 2012 Aihemäärittely - Sokoban}
\author{Mika Viinamäki}
\maketitle

\section{Aiheen kuvaus}
Sokoban on melko yksinkertainen, ruudukkopohjainen puzzle-peli, jossa ruudukossa sijaitsevaa pelaajaa liikuttaen on tarkoitus työntää kaikki laatikot tiettyyn kohtaan ruudukkoa. Ruudukossa sijaitsee seiniä, joiden läpi ei voi liikkua tai työntää laatikoita. Pelaaja ei myöskään voi työntää kahta laatikkoa yhtäaikaa.
    
Pelistä löytyy kunnollinen kuvaus wikipediasta: \url{http://en.wikipedia.org/wiki/Sokoban}
    
\section{Suunnitellut ominaisuudet}

\subsection{Keskeisimmät ominaisuudet}

Eli ominaisuudet, jotka nyt ainakin tulevat.

\begin{itemize}
    \item Graafinen käyttöliittymä - todennäköisesti Slickillä
    \item Kaikki tavallisen sokobanin ominaisuudet
    \item Kenttien lataus tiedostosta
\end{itemize}

\subsection{Muut ominaisuudet}

Jos jaksaa/löytyy motivaatiota/ehtii -osasto.

\begin{itemize}
    \item Highscore-lista
    \item Useampi ohjattava hahmo - ruudukossa on siis useampi liikuteltava hahmo, ja pelaaja voi valita mitä niistä haluaa liikuttaa. Ei tarkoita siis moninpeliä. 
    \item Omaperäisiä tilejä tavallisten Sokoban-tilejen lisäksi
\end{itemize}

\end{document}
